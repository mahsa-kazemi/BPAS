The \texttt{ZeroDimensionalRegularChain} class handles the special
case of regular chains of dimension zero (where there are the same
number of equations as variables in the chain). The zero sets of
zero-dimensional regular chains are therefore sets of points. Because zero-dimensional
regular chains are particular kinds of regular chains, the \texttt{ZeroDimensionalRegularChain}
class inherits from the \texttt{RegularChain} class. This special case allows for
specialized algorithms that are more efficient than those needed for positive dimension, as
well as being of mathematical interest in its own right, as we discuss briefly below.
Specialized versions of \texttt{intersect} and
\texttt{regularize} are provided by
\texttt{ZeroDimensionalRegularChain}, which behave in precisely the same
way as the arbitrary dimension routines, except that the underlying regular
chain has \texttt{ZeroDimensionalRegularChain} type. The usage of these
routines is illustrated below.
